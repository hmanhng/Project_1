\documentclass[a4paper]{article}
% \usepackage{vntex}
%\usepackage[english,vietnam]{babel}
%\usepackage[utf8]{inputenc}

%\usepackage[utf8]{inputenc}
%\usepackage[francais]{babel}
\usepackage{a4wide,amssymb,epsfig,latexsym,array,hhline,fancyhdr}
\usepackage[normalem]{ulem}
%\usepackage{soul}
\usepackage[makeroom]{cancel}
\usepackage{amsmath}
\usepackage{amsthm}
\usepackage{listings}
\usepackage{multicol,longtable,amscd}
\usepackage{diagbox}%Make diagonal lines in tables
\usepackage{booktabs}
\usepackage{alltt}
\usepackage[framemethod=tikz]{mdframed}% For highlighting paragraph backgrounds
\usepackage{caption,subcaption}

\usepackage{lastpage}
\usepackage[lined,boxed,commentsnumbered]{algorithm2e}
\usepackage{enumerate}
\usepackage{color}
\usepackage{graphicx}							% Standard graphics package
\usepackage{array}
\usepackage{tabularx, caption}
\usepackage{multirow}
\usepackage{multicol}
\usepackage{rotating}
\usepackage{graphics}
\usepackage{geometry}
\usepackage{setspace}
\usepackage{epsfig}
\usepackage{tikz}
\usetikzlibrary{arrows,snakes,backgrounds}
\usepackage[unicode]{hyperref}
\hypersetup{urlcolor=blue,linkcolor=black,citecolor=black,colorlinks=true}
%\usepackage{pstcol} 								% PSTricks with the standard color package

\usepackage[normalem]{ulem}

\newtheorem{theorem}{{\bf Định lý}}
\newtheorem{property}{{\bf Tính chất}}
\newtheorem{proposition}{{\bf Mệnh đề}}
\newtheorem{corollary}[proposition]{{\bf Hệ quả}}
\newtheorem{lemma}[proposition]{{\bf Bổ đề}}
\theoremstyle{definition}
\newtheorem{exer}{Bài toán}

\def\thesislayout{	% A4: 210 × 297
	\geometry{
		a4paper,
		total={160mm,240mm},  % fix over page
		left=30mm,
		top=30mm,
	}
}
\thesislayout

%\usepackage{fancyhdr}
\setlength{\headheight}{40pt}
\pagestyle{fancy}
\fancyhead{} % clear all header fields
\fancyhead[L]{
 \begin{tabular}{rl}
    \begin{picture}(25,15)(0,0)
    \put(0,-12){\includegraphics[width=10mm, height=10mm]{Images/soict.png}}
    %\put(0,-8){\epsfig{width=10mm,figure=hcmut.eps}}
   \end{picture}&
	%\includegraphics[width=8mm, height=8mm]{hcmut.png} & %
	\begin{tabular}{l}
		\textbf{\bf \ttfamily Trường Công Nghệ Thông Tin \& Truyền Thông}\\
		\textbf{\bf \ttfamily Công Nghệ Thông Tin Việt Pháp}
	\end{tabular}
 \end{tabular}
}
\fancyhead[R]{
	\begin{tabular}{l}
    \textbf{\LaTeX}
	\end{tabular}  }
\fancyfoot{} % clear all footer fields
\fancyfoot[L]{\scriptsize \ttfamily Báo cáo project 1}
\fancyfoot[R]{\scriptsize \ttfamily Trang {\thepage}/\pageref{LastPage}}
\renewcommand{\headrulewidth}{0.3pt}
\renewcommand{\footrulewidth}{0.3pt}


%%%
\setcounter{secnumdepth}{4}
\setcounter{tocdepth}{3}
\makeatletter
\newcounter {subsubsubsection}[subsubsection]
\renewcommand\thesubsubsubsection{\thesubsubsection .\@alph\c@subsubsubsection}
\newcommand\subsubsubsection{\@startsection{subsubsubsection}{4}{\z@}%
                                     {-3.25ex\@plus -1ex \@minus -.2ex}%
                                     {1.5ex \@plus .2ex}%
                                     {\normalfont\normalsize\bfseries}}
\newcommand*\l@subsubsubsection{\@dottedtocline{3}{10.0em}{4.1em}}
\newcommand*{\subsubsubsectionmark}[1]{}
\makeatother

\everymath{\color{blue}}%make in-line maths symbols blue to read/check easily

\sloppy
\captionsetup[figure]{labelfont={small,bf},textfont={small,it},belowskip=-1pt,aboveskip=-9pt}
%space remove between caption, figure, and text
\captionsetup[table]{labelfont={small,bf},textfont={small,it},belowskip=-1pt,aboveskip=7pt}
%space remove between caption, table, and text

%\floatplacement{figure}{H}%forced here float placement automatically for figures
%\floatplacement{table}{H}%forced here float placement automatically for table
%the following settings (11 lines) are to remove white space before or after the figures and tables
%\setcounter{topnumber}{2}
%\setcounter{bottomnumber}{2}
%\setcounter{totalnumber}{4}
%\renewcommand{\topfraction}{0.85}
%\renewcommand{\bottomfraction}{0.85}
%\renewcommand{\textfraction}{0.15}
%\renewcommand{\floatpagefraction}{0.8}
%\renewcommand{\textfraction}{0.1}
\setlength{\floatsep}{5pt plus 2pt minus 2pt}
\setlength{\textfloatsep}{5pt plus 2pt minus 2pt}
\setlength{\intextsep}{10pt plus 2pt minus 2pt}

\thesislayout

\definecolor{codegreen}{rgb}{0,0.6,0}
\definecolor{codegray}{rgb}{0.5,0.5,0.5}
\definecolor{codepurple}{rgb}{0.58,0,0.82}
\definecolor{backcolour}{rgb}{0.95,0.95,0.92}

\lstdefinestyle{mystyle}{
    backgroundcolor=\color{backcolour},
    commentstyle=\color{codegreen},
    keywordstyle=\color{magenta},
    numberstyle=\tiny\color{codegray},
    stringstyle=\color{codepurple},
    basicstyle=\ttfamily\footnotesize,
    breakatwhitespace=false,
    breaklines=true,
    captionpos=b,
    keepspaces=true,
    numbers=left,
    numbersep=5pt,
    showspaces=false,
    showstringspaces=false,
    showtabs=false,
    tabsize=2
}

\lstset{style=mystyle}

\begin{document}

\begin{titlepage}
\begin{center}
ĐẠI HỌC BÁCH KHOA HÀ NỘI \\
TRƯỜNG CÔNG NGHỆ THÔNG TIN \& TRUYỀN THÔNG \\
CÔNG NGHỆ THÔNG TIN VIỆT PHÁP
\end{center}

\vspace{1cm}

\begin{figure}[h!]
\begin{center}
\includegraphics[width=3cm]{Images/hust.png}
\end{center}
\end{figure}

\vspace{1cm}


\begin{center}
\begin{tabular}{c}
\multicolumn{1}{c}{\textbf{{\Large PROJECT 1}}}\\
~~\\
\hline
\\

\textbf{\large Truy xuất dữ liệu lớn với bộ nhớ nhỏ}
\\
\hline
\end{tabular}
\end{center}

\vspace{1.5cm}

\begin{table}[h]
\begin{tabular}{rrl}
\hspace{4 cm} & GVHD: & Nguyễn Đức Anh\\
& SV thực hiện: & Nguyễn Hữu Mạnh -- 20205213 \\
\end{tabular}
\end{table}
\vspace{2cm}
\begin{center}
{\footnotesize Hà Nội, Tháng 1/2024}
\end{center}
\end{titlepage}

%\thispagestyle{empty}
\tableofcontents
\newpage
\phantomsection

\section{Problem Statement}
Tạo một chương trình truy xuất dữ liệu lớn trên bộ nhớ kích thước nhỏ:
\begin{enumerate}
\item Tạo dữ liệu nhân tạo bằng cách: Tạo 3,000,000 (3 triệu) mảng interger, kích thước mỗi mảng là một giá trị nguyên ngẫu nhiên từ 100 tới 10000. Lưu trữ tuỳ ý (Gợi ý, lưu trữ qua file).
\item Tạo một hàm truy xuất ngẫu hiên tới mảng thứ i trong 3 triệu mảng trên (Gợi ý, cần lưu thêm giá trị bổ trợ để truy xuất tốt hơn).
\end{enumerate}
\textbf{Giới hạn bộ nhớ RAM: Dưới 500 MB.}

\section{Design}
\begin{enumerate}[]
  \item Lưu trữ
  \begin{enumerate}[-]
    \item Lưu dữ liệu vào file ở dạng nhị phân để bảo toàn dữ liệu.
    \item Lưu vị trí của mỗi mảng vào 1 file riêng để truy xuất.
  \end{enumerate}
  \item Truy xuất
  \begin{enumerate}[-]
    \item Sử dụng file lưu vị trí để truy cập đến mảng nhanh.
    \item In ra mảng đó.
  \end{enumerate}
\end{enumerate}
\texttt{ Để có thể giới hạn bộ nhớ, ta sử dụng java -Xmx500m}

\section{Implementation}
\subsection{Generate Arrays}
\begin{enumerate}[1.]
  \item Tạo biến lưu tổng số \texttt{array} và lưu giới hạn \texttt{elements} trong \texttt{array}
  \lstinputlisting[language=Java, firstline=10, lastline=11]{src/GenerateArrays.java}
  \item Thực hiện ghi vào 2 file \textbf{data.dat} và \textbf{pos.dat}\\
  File \textbf{data.dat} lưu dữ liệu ở dạng nhị phân\\
  File \textbf{pos.dat} vị trí của từng \texttt{array} ở dạng nhị phân
  \lstinputlisting[language=Java, firstline=13, lastline=18]{src/GenerateArrays.java}
  Khai báo 1 biến \texttt{pos} để lưu vị trí của từng \texttt{array}\\
  Ta cần biến \texttt{pos} là \texttt{long} vì mỗi \texttt{array} có 100 đến 10000 \texttt{elements} nên 3 triệu mảng sẽ có tổng số \texttt{elements} $>$ max value \texttt{int} có thể lưu là $2^{31}$
  \lstinputlisting[language=Java, firstline=19, lastline=19]{src/GenerateArrays.java}
  Lặp qua từng \texttt{array}
  \lstinputlisting[language=Java, firstline=21, lastline=21]{src/GenerateArrays.java}
  Khai báo biến \texttt{size} là số \texttt{elements} của \texttt{array i}. Biến \texttt{size} random trong khoảng \texttt{rangeArr}
  \lstinputlisting[language=Java, firstline=22, lastline=22]{src/GenerateArrays.java}
  Chuyển value \texttt{pos} sang \texttt{byte} và ghi vào file \textbf{pos.dat}
  \lstinputlisting[language=Java, firstline=24, lastline=26]{src/GenerateArrays.java}
  Lặp qua từng \texttt{element} trong \texttt{array}, lấy giá trị của mỗi \texttt{element} là random 1 số \texttt{int}\\
  Chuyển value \texttt{element} sang \texttt{byte} và ghi vào file \textbf{data.dat}
  \lstinputlisting[language=Java, firstline=28, lastline=32]{src/GenerateArrays.java}
  Sau khi ghi xong ta sẽ lưu thêm 1 value \texttt{pos} cuối cùng nữa để tính đươc số \texttt{element} của \texttt{array} cuối cùng
  \lstinputlisting[language=Java, firstline=34, lastline=37]{src/GenerateArrays.java}

  \item Hàm chuyển \texttt{int} hoặc \texttt{long} sang \texttt{byte}
  \lstinputlisting[language=Java, firstline=49, lastline=55]{src/GenerateArrays.java}
\end{enumerate}
\textbf{Code hoàn chỉnh}
  \lstinputlisting[language=Java]{src/GenerateArrays.java}

\subsection{Access Arrays}
\begin{enumerate}[1.]
  \item Nhập vào thứ tự của \texttt{array} muốn truy xuất
  \lstinputlisting[language=Java, firstline=8, lastline=11]{src/AccessArrays.java}
  \item Thực hiện truy xuất \texttt{array}
  \lstinputlisting[language=Java, firstline=13, lastline=13]{src/AccessArrays.java}
  \item Hàm truy xuất \texttt{getArray}\\
  Đọc dữ liệu từ 2 file \textbf{data.dat} và \textbf{pos.dat}\\
  Ta sử dụng \texttt{RandomAccessFile} để có thể truy nhập file bằng \texttt{byte}.
  \lstinputlisting[language=Java, firstline=16, lastline=18]{src/AccessArrays.java}
  Thực hiện lấy vị trí của \texttt{array} cần truy xuất\\
  Dùng \texttt{seek} để có thể truy nhập file vào đúng vị trí cần truy nhập
  \lstinputlisting[language=Java, firstline=20, lastline=21]{src/AccessArrays.java}
  Đọc vị trí và kích thước của \texttt{array}
  \lstinputlisting[language=Java, firstline=22, lastline=23]{src/AccessArrays.java}
  Truy nhập vào vị trí của \texttt{array} cần truy xuất
  \lstinputlisting[language=Java, firstline=26, lastline=26]{src/AccessArrays.java}
  Dùng vòng lặp \texttt{for} để đọc từng \texttt{element} của \texttt{array} cần truy xuất và in ra
  \lstinputlisting[language=Java, firstline=27, lastline=30]{src/AccessArrays.java}
\end{enumerate}
\textbf{Code hoàn chỉnh}
  \lstinputlisting[language=Java]{src/AccessArrays.java}

\subsection{Code}
\textbf{\href{https://github.com/hmanhng/Project_1}{Souce code toàn bộ của project_1}}

\section{Testing}
\subsection{Compile Code}
\item Thư mục \textbf{src/} chứa source code, thư mục \textbf{bin/} chứa các file \texttt{.class} và các file dữ liệu.
\item Giả sử ta đang ở thư mục cha của thu mục \textbf{src/} và thư mục \textbf{bin/}
\begin{mdframed}[backgroundcolor=backcolour, linecolor=backcolour]
  \texttt{javac -d ./bin src/*.java}
\end{mdframed}
Vào thư mục \texttt{bin}
\begin{mdframed}[backgroundcolor=backcolour, linecolor=backcolour]
  \texttt{cd ./bin}
\end{mdframed}
\subsection{Generate Arrays}
\begin{mdframed}[backgroundcolor=backcolour, linecolor=backcolour]
  \texttt{java GenerateArrays}
\end{mdframed}
Sau khi chạy ta được 2 file \textbf{data.dat} và \textbf{pos.dat} được lưu ở cùng thư mục \texttt{bin}
\subsection{Access Arrays}
\item Chạy AccessArrays với flag \texttt{-Xmx500m} để giới hạn bộ nhớ còn $500Mb$
\begin{mdframed}[backgroundcolor=backcolour, linecolor=backcolour]
  \texttt{java -Xmx500m AccessArrays\\
    Input index: // nhập 1 số i bất kì}
\end{mdframed}
$>$ In ra console \texttt{array i} cần truy xuất
\item Thời gian truy xuất với 3 triệu mảng, mỗi mảng từ 100 đến 10000 nghìn được thực hiện chỉ tính bằng mili giây, ta có thể kiểm tra bằng
\begin{mdframed}[backgroundcolor=backcolour, linecolor=backcolour]
  \texttt{time java -Xmx500m AccessArrays}
\end{mdframed}
\item lệnh này sẽ in ra thời gian thực hiện câu lệnh.


\subsection{Test}
\item Vì số lượng \texttt{array + element} quá lớn nên không thể kiểm tra được truy xuất có chính xác không nên ta sẽ tạo 1 file \textbf{test.java} với số \texttt{array} 100000, mỗi array từ 10-20 \texttt{element} để có thể kiểm tra việc truy xuất có đúng không
  \lstinputlisting[language=Java, firstline=12, lastline=13]{src/test.java}
  Ghi dữ liệu dạng text vào file \textbf{data.txt}
  \lstinputlisting[language=Java, firstline=15, lastline=15]{src/test.java}
  Trong vòng lặp \texttt{for} khi thực hiện random \texttt{element} và ghi vào \textbf{data.dat} ta cũng sẽ lưu dữ liệu vào \textbf{data.txt}, mỗi \texttt{array} sẽ được lưu vào 1 dòng
  \lstinputlisting[language=Java, firstline=31, lastline=38]{src/test.java}
\textbf{Code hoàn chỉnh}
  \lstinputlisting[language=Java]{src/test.java}
\item Biên dịch tương tự giống như trên
\item Tạo dữ liệu nhỏ hơn với 100000 \texttt{arrays}, mỗi \texttt{array} từ 10->20 \texttt{element}
\begin{mdframed}[backgroundcolor=backcolour, linecolor=backcolour]
  \texttt{java test}
\end{mdframed}
\item Sau khi chạy ta được 3 file \textbf{data.dat}, \textbf{pos.dat} và \textbf{data.txt} lưu dữ liệu ở dạng văn bản có thể đọc, mỗi mảng được lưu 1 dòng cách nhau bởi " ". Được lưu ở cùng thư mục \texttt{bin}
\item Truy xuất thử 1 \texttt{array} bất kì, ví dụ với \texttt{array} thứ \texttt{100000}
\begin{mdframed}[backgroundcolor=backcolour, linecolor=backcolour]
  \texttt{java -Xmx500m AccessArrays\\
    Input index: 100000}
\end{mdframed}
$>$ In ra console \texttt{array 100000} cần truy xuất
\item Để kiểm tra truy xuất có chính xác không ta kiểm tra file \textbf{data.txt}
\item Ta sử dụng command \texttt{`sed'} để in ra màn hình \texttt{array 100000}
\begin{mdframed}[backgroundcolor=backcolour, linecolor=backcolour]
  \texttt{sed -n '100000p' data.txt}
\end{mdframed}
Lệnh này sẽ in ra dòng thứ \texttt{100000} trong file \textbf{data.txt}
\item $>$ Đối chiếu kết quả của 2 lệnh thấy giống nhau $=>$ truy xuất chính xác.

\section{Conclude}
\begin{enumerate}[]
  \item Qua bài trên em đã học được cách tổ chức lưu trữ 1 dữ liệu lớn để có thể tối ưu trong việc truy xuất.
  \item Cách để có thể truy xuất vào 1 vị trí bất kì trong 1 file dữ liệu lớn trong thời gian ngắn gần như bằng \texttt{0} với bộ nhớ cực kì nhỏ.
\end{enumerate}
\end{document}
